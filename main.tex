\documentclass[12pt]{article}

%DEFINIZIONE DEI PACCHETTI GENERICI
\usepackage[a4paper,top=3.5cm,bottom=2.5cm,left=3cm,right=3.5cm]{geometry}
\usepackage{times}
\usepackage{titlesec}
%\usepackage{lipsum}
\usepackage{titletoc}

%DEFINIZIONE DEI PACCHETTI MATEMATICI
\usepackage{amsmath}
\usepackage{amsfonts}
\usepackage{amssymb}
\usepackage{amsthm}

%STILE DELLE PROPOSIZIONI
\theoremstyle{plain}

% Intestazioni in ingelse
\newtheorem{thm}{Theorem}[section]
\newtheorem{lem}[thm]{Lemma}
\newtheorem{prop}[thm]{Proposition}
\newtheorem{cor}[thm]{Corollary}
\newtheorem{defn}[thm]{Definition}
\newtheorem{rmk}[thm]{Remark}
\newtheorem{ex}[thm]{Example}
\newtheorem{prob}[thm]{Problem}
\newtheorem*{notat*}{Notation}
\newtheorem*{quest*}{Question}

%STILE DELLE SEZIONI
\titleformat{\section}
{\normalfont \scshape \centering}{\thesection.}{0,5em}{}
\titlespacing*{\section}{0pt}{50pt}{0.5cm}

\titleformat{\subsection}
{\normalfont \bfseries }{ \normalfont \thesubsection.}{0,5em}{}
\titlespacing*{\subsection}{0pt}{50pt}{0.5cm}

\titlecontents{section}[0cm]{}
{\normalfont \thecontentslabel. \enspace}
{\hspace*{-5.3em}}
{ \hfill \normalfont \contentspage}

\titlecontents{subsection}[0cm]{}
{\normalfont \thecontentslabel. \enspace}
{\hspace*{-5.3em}}
{ \hfill \normalfont \contentspage}

%OPZIONI PER LA BIBLIOGRAFIA
\usepackage[
backend=bibtex,
style=alphabetic,
]{biblatex}
%\addbibresource{bibliography.bib}
%\AtNextBibliography{\small}


%INIZIO DEL DOCUMENTO
\begin{document}
	
	\fontsize{12pt}{0pt}
	\begin{center}
		\textbf{TITOLO}
	\end{center}
	
	\tableofcontents

	%INIZIA A SCRIVERE QUI
    \section{Background}
    \subsection{Chow ring}
    Let $\mathcal{L}$ be a geometric lattice and $\mathcal{G} \subset \mathcal{L}$ a building set. We we can associate to the couple $(\mathcal{L}, \mathcal{G})$ a ring called Chow ring $\mathcal{A}(\mathcal{L}, \mathcal{G})$. \\
    We can define the chow ring of the couple $(\mathcal{L}, \mathcal{G})$ as
    \[ \frac{S}{I + J},\]
    where $S = \mathbb{Z}[x_F]_{F \in \mathcal{L}}$, $I$ is generated by the elements
    \[
    x_{F_1} \cdots x_{F_n} \text{ such that } \{F_1, \dots, F_n\} \subset \mathcal{G} \text{ is not a $\mathcal{G}$-nested set}
    \]
     and $J$ by the linear elements
    \[ \sum_{a \le F \in \mathcal{G}} x_F \text{ for each atom $a \in \mathcal{L}$ }. \]
    
    \begin{rmk}
        The chow ring si a graded algebra, so we can decompose it in a direct sum of subspaces
        \[ \mathcal{A}(\mathcal{L}, \mathcal{G}) = \bigoplus_{k \ge 0} A^k. \]
        In particular the superior limit of this direct sum is $r$ where $rk(\mathcal{L}) = r+1$,
        \[ \mathcal{A}(\mathcal{L}, \mathcal{G}) = \bigoplus_{k = 0}^r A^k. \]
    \end{rmk}
    
    

    \subsection{Feichtner-Yuzvinsky monomial basis}    
    In this section our goal is to describe a monomial basis for the Chow ring.
    To introduce this base we need to define some machinery first.

    Let $\mathcal{L}$ be an atomistic lattice and define $Int(\mathcal{L}) := \{ (F, F') \in \mathcal{L} \times \mathcal{L} \, : \, F \le F' \}$. Then the function \[ d: Int(\mathcal{L}) \longrightarrow \mathbb{N} \] is well-defined as
    \[ d( F, F') := \min\{ d \in \mathbb{N} \, : \, F' = F \vee a_1 \vee \dots \vee a_d\text{, for some atoms $a_1, \dots a_d \in \mathcal{L}$} \}.\]

    If $\mathcal{L}$ is an atomistic semi-modular lattice, then the function $d$ can be duduced by the rank function:
    \begin{equation} \label{eq:d(ff')}
        d(F, F') = rk(F') - rk(F).
    \end{equation}

    We define now a function called $m_N$, where $N \subseteq \mathcal{G}$ is a $\mathcal{G}$-nested set, using the distance function $d$.

    \begin{defn}
        Let $\mathcal{L}$ a lattice, $\mathcal{G} \subset \mathcal{L}$ a building set and $N \subset \mathcal{G}$ a $\mathcal{G}$-nested set. Given $F \in N$ define:
        \[ m_N(F) := d(\vee N_{<F}, F).\]
    \end{defn}

    If $\mathcal{L}$ is a geometric lattice then, by (\ref{eq:d(ff')}) we have
    \begin{equation} \label{useful_id}
        m_N(F) = rk(F) - rk(\vee N_{<F}).
    \end{equation}

    \begin{lem} \label{lem:usefulineq}
        Let $\mathcal{L}$ a geometric lattice, $\mathcal{G}$ a building set and $N \subset \mathcal{G}$ a $\mathcal{G}$-nested set, then:
        \[ rk(\vee N) = \sum_{F \in N} m_N(F).\]
    \end{lem}
    
    We now define a particular type of monomial order, the $FY$-monomial order.
    \begin{defn}
    The FY-monomial order on the ring $S$ is the one based on the linear order of the variable, i.e. 
    \[ X_F > X_{F'} \text{ if } F < F'.\] 
    \end{defn}
    
\begin{thm} \label{thm:GrobB}
    Let $\mathcal{L}$ be a finite atomic lattice, and $\mathcal{G} \subset \mathcal{L}\setminus\{\hat{0}\}$ a building set. Then for any choice of an FY-monomial order on the ring $S$, the ideal $I+J$ presenting $\mathcal{A}(\mathcal{L}, \mathcal{G}) = S/I+J$ has a monic Grobner basis consisting of the following elements:
    \begin{enumerate}
        \item all $x_{F_1} x_{F_2} \cdots x_{F_n}$ for all $\{ x_{F_1}, \dots, x_{F_n} \} \subset \mathcal{G}$ which are not 
            $\mathcal{G}$-nested;
            
        \item for each $\mathcal{G}$-nested set $N$ with maximal elements $\{ F_1, \dots , F_l \}$, and for each 
            $F \in \mathcal{G}$ with $F > \vee N$ the product:
            \[ x_{F_1}x_{F_2} \cdots x_{F_l} \cdot \left( \sum_{F' \ge F} x_{F'} \right)^{d(\vee N, F)} \]
            with his initial $\prec$-term which is:
            \[ x_{F_1} \cdots x_{F_l} x_F^{d(\vee N, F)}.\]
        \end{enumerate}
    \end{thm}

    As a corollary of this theorem we can provide a monomial basis.
    
    \begin{cor}
        Let $\mathcal{L}$ be a finite atomic lattice, and $\mathcal{G} \subset \mathcal{L}\setminus\{\hat{0}\}$ a building set. Then the Chow ring $\mathcal{A}(\mathcal{L}, \mathcal{G})$ is free as a $\mathbb{Z}$-module, with the $\prec$-standard monomial $\mathbb{Z}$-basis given by the set of FY-monomials:
            \begin{equation}
                FY := \{ x_{F_1}^{m_1} \cdots x_{F_l}^{m_l} \, : \, N = \{F_1, \dots, F_l \} \text{ is }\mathcal{G} \text{-nested, and } 0 < m_i < m_{N}(F_i) \}.
            \end{equation}    
    \end{cor}

    \begin{notat*} 
        We denote with $FY^k$ the monomial in $FY$ with degree k.
    \end{notat*}
    
    \subsection{Symmetric chain decomposition}
    We want now to cover the basis $FY$ with chains.

    Define the function
    \[ supp_+ : FY \longrightarrow \mathcal{N}(\mathcal{L}, \mathcal{G})\]
    called extended support map, defined as follows: if $m_i \ge 1$ for $1 \le i \le l$, then
    \[ supp_+(x_{F_1}^{m_1} \cdots x_{F_l}^{m_1l}) := \{F_1, \dots, F_l \} \cup \{E\};\]
    observe that $supp_+$ is well-defined because $\mathcal{N}(\mathcal{L}, \mathcal{G})$ is a simplicial complex, i.e.
    \[ \{F_1, \dots, F_l\} \in \mathcal{N}(\mathcal{L}, \mathcal{G}) \implies \{F_1, \dots, F_l\} \cup \{E \} \in \mathcal{N}(\mathcal{L}, \mathcal{G}). \]
    where $\mathcal{N}(\mathcal{L}, \mathcal{G})$ is the set of $\mathcal{G}$-nested sets in $\mathcal{L}$.

    We want to use the function $supp_+$ to cover the set $FY$ using its fiber. Consider $N^+ =\{F_1, \dots, F_l, E\} \subset \mathcal{G}$ a $\mathcal{G}$-nested set in the image of $supp_+$ then the fiber has the following description:
    \[ supp_+^{-1} (N^+) = \{x_{F_1}^{m_1} \cdots x_{F_l}^{m_l} \cdot x_E^{m_{l+1}}  \}\]
    where the exponents satisfy the inequalities:
    \[ 1 \le m_i \le m_{N^+}(F_i) - 1, \text{ for } 1 \le i \le l;\]
    \[ 0 \le m_{l+1} \le m_{N^+}(E)- 1\]

    \begin{rmk}
        The degree of the monomials in $supp_+^{-1}(N^+)$ lie in the range \[ \left[ l, r-l \right] \subset \mathbb{N}.\]
    \end{rmk}

    
    \begin{rmk}
        The fibers of $supp^+$ are always disjoint.
    \end{rmk}

    Once we covered $FY$ with the fibers of $supp^+$, we want to cover every fibers with chains. In particular we cover them with symmetric chains.
    
    We briefly introduce here these chains and the symmetric chain decomposition.

    \begin{defn}
        Let $(\mathcal{P}, \le)$ a ranked poset with rank function $rk : \mathcal{P} \longrightarrow \mathbb{N}$, we say that the elements $x_1, \dots, x_h \in \mathcal{P}$ form a symmetric chain if :
        \begin{enumerate}
            \item $x_{i+1} \gtrdot x_{i}$ for all $i < h$;
            \item $rk(x_1) + rk(x_h) = rk(\mathcal{P}).$
        \end{enumerate}
        where $rk(\mathcal{P})$ is the largest rank in $\mathcal{P}$.
    \end{defn}
    
    \begin{rmk}
        Suppose that $rk(\mathcal{P}) = k$, then consider $x_1, \dots, x_h \in \mathcal{P}$ a symmetric chain in $\mathcal{P}$. Then, by condition (2) of the above definition we have that $rk(x_1) = k - rk(x_h)$, so the length of the chain above the middle rank of the poset $\mathcal{P}$ is equal to the length of the chain under the middle rank of the poset $\mathcal{P}$.
    \end{rmk}
    
    \begin{defn}
        Let $(\mathcal{P}, \le)$ a ranked poset with rank function $rk : \mathcal{P} \longrightarrow \mathbb{N}$, a symmetric chain ($SCD$) of $\mathcal{P}$ is a set of disjoint symmetric chains that cover all the elements of $\mathcal{P}$.
    \end{defn}

    As a result, ordering the fibers of $supp^+$ by divisibility we can apply the symmetric chain decomposition to every fiber and obtain a decomposition of $FY$ in chains.
    
    \subsection{Hilbert series}
    Consider a graded algebra $A = \bigoplus_{k \ge 0} A^k$. The Hilbert series associated to $A$ is
    \[ HS(t) = \sum_{k \ge 0} \dim A^k \cdot t^k.\]

    \section{The polynomial $q(t)$}
    In this section we want to describe the Hilbert series associated to $\mathcal{A}(\mathcal{L}, \mathcal{G})$ by using symmetric chains.
    
    \subsection{Characterization of element's degree of a chain}
    In this section we want to establish a criterion to see if an element of degree $k$ belongs or not to a particular chain.
    \begin{notat*}
        Consider a chain $\gamma$. We denote with $\min(\gamma)$ the degree of the first element of $\gamma$ and with $\ell(\gamma)$ the length of $\gamma$, i.e. the number of element of $\gamma$.
    \end{notat*}
    
    Consider now a degree $k$ monomial $m$ in $FY$. What are the characteristic that the chain $\gamma$ need to have in order to contain $m$? let's consider some examples.

    \begin{ex}
        \begin{enumerate}
            \item[(1)] Consider $m=1$. This is the only element in $FY^0$. Since the chains are all disjoint there exist a unique chain $\bar\gamma$ that contains the element $m$.
            So the characteristic that need to have $\bar \gamma$ in order to contain $m$ is 
            \[ \min(\bar \gamma) = 0.\]

            \item[(2)] Consider $m \in FY^2$ and a chain $\gamma$. Since the minimum degree is $0$, we have three cases:
            \begin{enumerate}
                \item[(a)] $m$ is the minimum of $\gamma$.
                \item[(b)] $m$ is the second element of $\gamma$.
                \item[(c)] $m$ is the third element of $\gamma$.
            \end{enumerate}
        \end{enumerate}
    \end{ex}

    Thus the following lemma is valid.
    \begin{lem}
        Let $\gamma$ be a chain. Then
        \[\begin{matrix}
        \gamma \text{ contains an }  &  & \min(\gamma) = k, \text{ or } \\ 
         & \iff &  \\
        \text{ element of degree $k$ }&  &\min(\gamma) = d < k \text{ and } \ell(\gamma) > k-d
        \end{matrix}\]
    \end{lem}

    \begin{proof}
        Consider a chain $\gamma$. If $\min(\gamma) = k $, then the first element of $\gamma$ has degree $k$.
        If $\min(\gamma) = d < k$, and $\ell(\gamma) > k-d$, then $(k-d+1)-th$ element of $\gamma$ has degree $k$.
        
        Suppose now that $\min(\gamma) > k$, then in that case 
        \[ \forall m \in \gamma, \, \deg(m) > k\]
        and so $\gamma$ does not contain element of degree $k$.
    \end{proof}

    \begin{rmk}
        A a consequence we have that given a chain $\gamma$ with $\min(\gamma) = k$ and $\ell(\gamma) = d$, then the element of $\gamma$ has degree in the range
        \[ \left[ k, k+d-1 \right] \]
    \end{rmk}

    \subsection{Description of the dimension of $A^i$ with symmetric chain}
    In section 1.2 we provided a monomial basis for the chow ring $\mathcal{A}(\mathcal{L}, \mathcal{G})$, and we denoted with $FY^k$ the basis of $A^k$.
    In this section we want to describe $\dim A^i$ using symmetric chains.
    
    We first observe that $\dim A^i = |FY^i|$, where $|S|$ is the cardinality of the set $S$.
    We also denote by $SC$ the set of chains in the decomposition of $\mathcal{A}(\mathcal{L}, \mathcal{G})$.

    Thus, it is sufficient to describe $|FY^i|$ to give a representation of $\dim A^i$ with symmetric chains.
    By definition
    \[ FY^i = \{ m \in FY  \mid \deg(m) = i \}, \]
    so by the preceding lemma we have that a chains $\gamma$ contains an element of $FY^i$ if and only if 
    \[ \min(\gamma) = i, \text{ or } \min(\gamma) = d < i \text{ and } \ell(\gamma) > i-d\]
    thus
    \[ |FY^i| = | \left \{ \gamma \in SC \mid \min(\gamma) = i, \text{ or } \min(\gamma) < i \text{ and } \ell(\gamma) > i-\min(\gamma) \right \} |\]

    So the first results we obtained is
    \begin{align*}
        HS(t) =& \sum_{k \ge 0} \dim A^k \cdot t^k = \sum_{k \ge 0} |FY^k| \cdot t^k \\
        =&   \sum_{k \ge 0} | \left \{ \gamma \in SC \mid \min(\gamma) = k, \text{ or } \min(\gamma) < k \text{ and } \ell(\gamma) > k-\min(\gamma) \right \} | \cdot t^k
    \end{align*}

    \begin{rmk}
        Observe that the set \[ \left \{ \gamma \in SC \mid \min(\gamma) = k, \text{ or } \min(\gamma) < k \text{ and } \ell(\gamma) > k-\min(\gamma) \right \} \]
        can be described as a union of the two sets
        \[ \left \{ \gamma \in SC \mid \min(\gamma) = k \right \}  \]
        and
        \[ \left \{ \gamma \in SC \mid \min(\gamma) < k \text{ and } \ell(\gamma) > k-\min(\gamma) \right \}.\]
    
        We can also observe that these two sets are disjoint, indeed if $\gamma$ in in the intersection of these two sets, then $\min(\gamma) = k$ and $\min(\gamma) < k$, which is absurd. So they are disjoint.
        Thus the cardinality of these two sets adds up to give the cardinalty of the whole set
        \begin{align*}
            &| \left \{ \gamma \in SC \mid \min(\gamma) = k, \text{ or } \min(\gamma) < k \text{ and } \ell(\gamma) > k-\min(\gamma) \right \} | = \\
            &| \left \{ \gamma \in SC \mid \min(\gamma) = k \right \} | + |
            \left \{ \gamma \in SC \mid \min(\gamma) < k \text{ and } \ell(\gamma) > k-\min(\gamma) \right \} |
        \end{align*} 
    \end{rmk}
	
    \begin{notat*}
        $D(k) := \left \{ \gamma \in SC \mid \min(\gamma) = k, \text{ or } \min(\gamma) < k \text{ and } \ell(\gamma) > k-\min(\gamma) \right \}$
    \end{notat*}

    We want now to transform the summation over $k \ge 0$ in a summation over $\gamma \in SC$.
    To do that consider $\gamma \in SC$ such that $\min(\gamma) = k$ and $\ell(\gamma) = d$.
    
    \begin{quest*}
        What are the $k \in \mathbb{N}$ such that $\gamma \in D(k)$?
    \end{quest*}

    First we can say that $\gamma \in D(k)$ since $\min(\gamma) = k$.
    We want to find all the $x>k$ such that $\gamma \in D(x)$.
    Obviously if $x < k$ then $\gamma \notin D(x)$, in fact 
    \[ \gamma \in D(x) \iff \min(\gamma) = x, \text{ or } \min(\gamma)<x \text{ and } \ell(\gamma) > x-\min(\gamma),\]
    but $\min(\gamma) = k \neq x$, so we need $\min(\gamma) = k < x$, which is in contradiction with the hypothesis $x < k$.
    Therefore $x > k$. In this case we have
    \[ \gamma \in D(x) \iff  \min(\gamma) = x, \text{ or } \min(\gamma)<x \text{ and } \ell(\gamma) > x-\min(\gamma),\]
    but $\min(\gamma) \neq x$, so the condition restricts to 
    \[ \gamma \in D(x) \iff  \min(\gamma)<x \text{ and } \ell(\gamma) > x-\min(\gamma), \]
    the conduction $\min(\gamma) < x$ is satisfied by hypothesis, we want to find the $x$'s that satisfies the second condition:
    \[ d > x - k \iff x < d+k.\]
    We have proved the following Lemma.
    \begin{lem}
        Given $\gamma \in SC$,
        \[ \gamma \in D(x) \iff x \in \text{$\left [ k, k+d-1 \right ]$} \]
    \end{lem}

    So given a chain $\gamma \in SC$ what are the term it contributes in the Hilbert series?
    It contributes the terms
    \[ t ^{\min(\gamma)} + \dots + t^{\min(\gamma) + \ell(\gamma) -1} = t^{\min(\gamma)} 
    \left ( 1 + t + \dots + t^{\ell(\gamma) -1} \right).\]

    \begin{ex}
        AGGIUNGERE ESEMPIO (é sul foglio)
    \end{ex}
    
    Define the polynomial 
    \[ q(t) := \sum_{\gamma \in SC} t^{\min(\gamma)} \left ( 1 + t + \dots + t^{\ell(\gamma) -1} \right).\]
    
    Thus we proved the following theorem.
    \begin{thm} \label{T:main}
        \[HS(t) = q(t).\]
    \end{thm}

    \subsection{A proof in the inverse direction}

    We give here another proof of Theorem \ref{T:main} starting from $q(t)$.
    \begin{proof} [Proof of Theorem \ref{T:main}]
    Consider the polynomial 
    \[ q(t) = \sum_{\gamma \in SC} t^{\min(\gamma)}(1 + \dots + t^{\ell(\gamma) -1}).\]
    \begin{notat*}
        Given $\gamma \in SC$, then denote with $\gamma_i$ the $i$-th element of the chain $\gamma$.
    \end{notat*}
    
    Now we associate to each $\gamma \in SC$ a polynomial given by
    \[ r_{\gamma}(t) = \sum_{\gamma_i \in \gamma} t^i,\]
    thus
    \[ q(t) = \sum_{\gamma \in SC} t^{\min(\gamma)} r_\gamma(t).\]
    So $r_\gamma(t)$ count the number of element in the chain $\gamma$, but we want to take in account the degree of every element of $\gamma$, so we need to translate all the exponent in $r_\gamma(t)$ to match the degree of the element of $\gamma$. To do that we multiply $r_\gamma(t)$ by $t^{\min(\gamma)}$.

    Indeed, for every $i$ we have that $\deg(\gamma_i) = i - 1 + \min(\gamma)$ and thus taken $\gamma \in SC$, the element $\gamma_i$, contributes to $q(t)$ with $t^{i-1 + \min(\gamma)}$.
    
    Therefore given $k \in \mathbb{N}$, a chain $\gamma \in SC$ contributes with $t^k$ if and only if $\exists x \in \gamma$ such that $\deg(x) = k$.
    Finally
    \begin{align*}
        q(t) &= \sum_{\gamma \in SC} t^{\min(\gamma)}(1 + \dots + t^{\ell(\gamma) -1}) =\\
        &= \sum_{\gamma \in SC} t^{\min(\gamma)} + \dots + t^{\ell(\gamma) -1 + \min(\gamma)} = \\
        &= \sum_{k \ge 0} |\left\{ \gamma \in SC \mid \exists x \in \gamma : \deg(x) = k \right\}| t^k =\\
        &= \sum_{k \ge 0} |FY^k| t^k = \sum_{k \ge 0} \dim A^k \cdot t^k = HS(t).
    \end{align*}
        
    \end{proof}

    \section{Generalization of $q(t)$ by adding a variable $u$}

    Define the polynomial
    \[ p(t,u) = \sum_{\gamma \in SC} u^{\min(\gamma)} (1+\dots+t^{\ell(\gamma) -1}).\]
    It's clear that setting $u=t$ we can recover $q(t)$
    \[ p(t,t) = \sum_{\gamma \in SC} t^{\min(\gamma)}(1+\dots+t^{\ell(\gamma)-1}) = q(t).\]
    
    We now look at some specialization of the polynomial $p(t,u)$.
    \subsection{Case $u = 1$}
    In this case we have
    \[ p(t,1) = \sum_{\gamma \in SC} 1^{\min(\gamma)}(1+\dots+t^{\ell(\gamma) -1}) = \sum_{\gamma \in SC} 1+\dots+t^{\ell(\gamma) -1}.\]

    Here the polynomial is counting only the elements without considering the grade of the elements.
    Thus a chain $\gamma \in SC$ such that $\ell(\gamma) = k$ contributes to the polynomial with
    \[ 1+\dots+t^{k-1},\]
    so we have
    \[ p(t,1) = \sum_{k\ge0} L(k) t^k,\]
    where $L(k) = \{ \gamma \in SC : \ell(\gamma) > k\}$.

    \subsection{Case $t = 1$}
    In this case
    \[ p(1,u) = \sum_{\gamma \in SC} u^{\min(\gamma)} (1 + \dots + 1^{\ell(\gamma) -1}) = \sum_{\gamma \in SC} \ell(\gamma) u^{\min(\gamma)}\]

    \subsection{Case $t = 0$}
    In this case
    \[ p(0,u) =  \sum_{\gamma \in SC} u^{\min(\gamma)} (1 + \dots + 0^{\ell(\gamma) -1}) = \sum_{\gamma \in SC} u^{\min(\gamma)}.\]

    So we are counting only the chains in $SC$ with a given minimum. If a chain $\gamma \in SC$ is such that $\min(\gamma) = k$, then it contributes to the polynomial with $u^k$, so
    \[ p(0,u) = \sum_{k \ge 0} M(k)u^k, \]
    where $M(k) = \{\gamma \in SC : \min(\gamma) = k\}$.

    \subsection{Coefficients of $p(t,u)$}
    The last thing to do is to understand what are the coefficients of $p(t,u)$.
    Consider $\gamma \in SC$, then $\gamma$ contributes to the polynomial with
    \[ u^{\min(\gamma)} (1+ \dots + t^{\ell(\gamma)-1} ),\]
    thus a chain $\gamma$ contributes with the coefficients $u^kt^d$ if $\min(\gamma) = k$ and $\ell(\gamma) > d$.
    Therefore the coefficients of $u^kt^d$ is $|SC(k,d)|$ where
    \[ SC(k,d) = \{ \gamma \in SC : \min(\gamma) = k, \, \ell(\gamma) >d \}, \]
    giving
    \[ p(t,u) = \sum_{\gamma \in SC} |SC(k,d)|u^kt^d.\]

    \subsection{Some properties of SC(k,d)}
    Fix $k \in \mathbb{N}$, then we have
    \[ SC(k,0) \supseteq SC(k,1) \supseteq SC(k,2) \supseteq \dots\]

    This holds because if we fix $\min(\gamma) = k$, and we consider $d > f$ then we have
    \[ \gamma \in SC(k,d) \iff \ell(\gamma) > d \implies \ell(\gamma) > f \iff \gamma \in SC(k,f),\]
    so
    \[ f < d \implies SC(k,d) \subseteq SC(k,f).\]

    This sequence of inclusion is not infinite since a chain can't have a length greater than the maximum grade of the Chow ring $r$, so
    \[ d > r \implies SC(k,d) = \emptyset \, , \, \forall k \in \mathbb{N}.\]

    So in the end $\forall k \in \mathbb{N}$ we have the sequence of inclusion
    \[ SC(k,0) \supseteq SC(k,1) \supseteq \dots \supseteq SC(k,r-2) \supseteq SC(k,r-1).\]

    We now observe that the length of a chain depends on it's minimum. In fact if $\min(\gamma) = k$, since the chain is symmetric with respect to it's middle element, then the maximum grade of $\gamma$ would be at most $r-k$ and its lenth will be at most $r-2k$. So for every $k \in \mathbb{N}$, we have
    \[ SC(k,0) \supseteq SC(k,1) \supseteq \dots \supseteq SC(k, r-2k-1).\]
    
    Consider now $k=0$. Then, since we have only one chain that contains the element of degree $0$, denote ti by $\bar\gamma$, we have that
    \[ |SC(0,d)| = 1, \, \forall d <\ell(\bar\gamma).\]
    
    %STAMPA DELLA BIBLIOGRAFIA
	%\printbibliography[title=References and Bibliography]
	
\end{document}